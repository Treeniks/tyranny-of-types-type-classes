\documentclass[sigconf,review,svgnames,dvipsnames,nonacm]{acmart}
\acmConference[Types 2022]{The Tyranny of Types: Curse or Blessing?}{11 - 14 July, 2022}{Munich}
\setcopyright{none}
\copyrightyear{2022}
\acmYear{2022}
\acmPrice{0}
\bibliographystyle{ACM-Reference-Format}
\usepackage{amsmath}
%%%%%%%%%%%%%%%% DO NOT EDIT THE COMMANDS ABOVE %%%%%%%%%%%%%%%%%%%%%

%%%%%%%%%%%%%%%% PACKAGES %%%%%%%%%%%%%%%%%%%%%
\usepackage[english]{babel}
\usepackage{blindtext}
\usepackage{minted}
\usepackage{hyperref}
\usepackage{microtype}
% \usepackage{fancyvrb}

\VerbatimFootnotes


%%%%%%%%%%%%%%%% DOCUMENT %%%%%%%%%%%%%%%%%%%%%
\begin{document}
\title[Comparing Type Classes, Traits and Modules]{Comparing Haskell's Type Classes to Rust's Traits and Ocaml's Modules}
% \author{}
% \affiliation{}
\author{Thomas Lindae}
% \authornotemark[1]
\email{thomas.lindae@tum.de}
\affiliation{%
  \institution{Technical University of Munich}
  % \streetaddress{}
  % \city{Munich}
  % \state{Bavaria}
  \country{Germany}
}

% ---- Abstract ----
\begin{abstract}
%%%%%%%% Write your abstract %%%%%%%%%%%%%%%
There are many ways to abstract functionality in programming languages. One important idea of abstraction is the notion of interfaces defining functionality for abstract types. Modern languages offer different approaches to such interfaces, some of which with a focus on data abstraction, while others with a focus on supporting ad-hoc polymorphism. We will first see the differences between parametric and ad-hoc polymorphism. Afterwards we will look at three interfacing approaches: Haskell type classes, the type class-inspired Rust traits and Ocaml's ML-inherited module system, and find them to have many similarities.
\end{abstract}
\maketitle

% ---- Main Text of the Document ----
%%%%%%%% If your text is in separate files (recommended), input them here %%%%%%%%%%%%%%%

\section{Introduction}

Programmers are lazy. That is why they tend to spend a lot of time coming up with and developing programming language concepts that allow them to write less code for the same things, and then let the compiler do the rest of the work.

Few things convey that idea better than the concept of polymorphism, a term describing the reusing of functions and methods for varying data types.

There is many different forms of polymorphism, and in this paper we will briefly look at two of them: Parametric and ad-hoc polymorphism.

Ad-hoc polymorphism is primarily used by object oriented programming languages, whereas parametric polymorphism is more well known by its use in functional languages. However, as languages have matured and gained features, most modern languages have their own concepts for both.

Afterwards we will look at one particular implementation of ad-hoc polymorphism: Type Classes. We will find that the concept of type classes also offers a way to define an interface on custom data types and thus compare them to other language interface concepts, in particular Rust Traits and Ocaml Modules.

\section{Types of Polymorphism}

As defined by Benjamin C. Pierce~\cite[Chapter~23.2]{pierce-types}, a polymorphic type system refers to a type system that allows a single piece of code to be used with multiple types. There are several varieties of polymorphism, of which we will cover two.

\subsection{Parametric Polymorphism}

Parametric polymorphism is the idea that a function's input types can be parameters, where the exact type inserted in these parameters is irrelevant for the function's semantics. That means that \emph{any} type can be inserted~\cite[Chapter~23.2]{pierce-types}.

This is best understood by a common example for parametric polymorphism, the length function for lists in functional languages. For this, we will look at how such a function could look like in OCaml:
\begin{minted}{ocaml}
let rec length l =
  match l with
  | [] -> 0
  | x::xs -> 1 + length xs
\end{minted}
The inferred type of \verb|length| would then be
\begin{minted}{ocaml}
val length : 'a list -> int
\end{minted}
where \verb|'a| is a type parameter. The important thing to understand about parametric polymorphism is that the actual value of \verb|'a| is irrelevant to how this function operates. This means that there must not be more than one implementation of \verb|length| for different concrete instantiations of \verb|'a|. Instead there is exactly one implementation and it can be called for all types of lists with the same semantics.

The reason this works is that we don't ever infer what the type of \verb|x| inside the function should be. To determine the length of a list, the type of the list's elements is irrelevant.

The compiler might still be free to compile different versions of this function, depending on what types the function is called with. Depending on how lists are implemented, small parts of the list might be laid out in memory as arrays~\cite{functional-lists}, which means that the act of getting the next value in the list requires the compiler to know the size of \verb|'a|. However, this is hidden from the programmer and not relevant for the language definition.

The problem with parametric polymorphism is that it is inherently useless for functions that should have different implementations for different types, and potentially no implementation for some types. The obvious example for this case would be operator overloading, such as a \verb|(+)| function that does addition.

In OCaml, this is solved by having a different operator for different data types, for example \verb|(+)| for \verb|int|s and \verb|(+.)| for \verb|float|s. The equality check function \verb|(=)| is handled special in OCaml, in that every type automatically implements it. Thus the function is parametrically typed as
\begin{minted}{ocaml}
val (=) : 'a -> 'a -> bool
\end{minted}
and the implementation of this function is handled by the compiler and runtime system. Because this is the type of the \verb|(=)| function, OCaml will not give you a compile time error if you happen to try to check two functions for equality, even though functions can not generally be checked for equality. OCaml can only throw an exception at runtime in such cases.
% CITATION?

\subsection{Ad-hoc Polymorphism}\label{ad-hoc-polymorphism}

Ad-hoc polymorphism is the idea of overloading functions for different types~\cite[Chapter~23.2]{pierce-types}. This means one can have one implementation for a specific set of types, but potentially a different implementation, including none, for a different set of types. This fixes the aforementioned problem of parametric polymorphism. Operator overloading in particular is a common use case for ad-hoc polymorphism. It allows to have different implementations of \verb|(+)|, both for \verb|int -> int -> int| and \verb|float -> float -> float|.

However, the exact implementation for ad-hoc polymorphism in a language is not necessarily trivial. If one were to write \verb|(+) x y| inside their code, the compiler must now be able to determine which \verb|(+)| should be called. It needs to figure this out from the types of \verb|x| and \verb|y|, whereas before it was always unambiguous what function \verb|(+)| refers to. While this is still reasonably simple when \verb|x| and \verb|y| are variables or constants with a defined type, once \verb|x| and \verb|y| are themselves parameters with polymorphic types, the amount of functions that need to be created is potentially exponential~\cite{type-classes-original}.

See for example an \verb|add2| function that is defined like so:
\begin{minted}{ocaml}
let add2 (x1, x2) (y1, y2) = (x1 + y1, x2 + y2)
\end{minted}
Assuming that \verb|(+)| is defined for both \verb|int| and \verb|float|, then there are four possible types for \verb|add2|:
\begin{minted}{ocaml}
val add2 : int * int -> int * int -> int * int
val add2 : int * float -> int * float -> int * float
val add2 : float * int -> float * int -> float * int
val add2 : float * float -> float * float
                                  -> float * float
\end{minted}
This can grow exponentially and becomes inefficient. Thus, implementations of ad-hoc polymorphism typically have clever ways of circumventing this kind of blowup.

One way is to use a system of dynamic dispatch as seen in Java, another is to make heavy use of higher order functions as seen in Haskell~\cite{type-classes-original}. Both approaches work on a similar idea: One must first define a sort of \emph{standard} for the \verb|(+)| function we can group the implemented types under. For the sake of example, we will call types that implement \verb|(+)| \emph{addable}. I.e. \verb|int| and \verb|float| are \emph{addable}s. Then we can built a system to tell the compiler that \verb|add2| is defined for parameters whose types are \emph{addable}s. Whenever we define a type as \emph{addable}, i.e. implement the \verb|(+)| function, we also create a dictionary with information about where to find the implemented function. Since we standardized the \verb|(+)| function under the \emph{addable} interface, these dictionaries will look the same for every \emph{addable} type. When calling \verb|add2|, all we need to do is also pass the appropriate dictionary to \verb|add2|. \verb|add2| then uses that dictionary to find the appropriate \verb|(+)| function and call it for the arguments. That way, only one implementation for \verb|add2| is needed to cover all \emph{addable} types and there is no exponential blowup. However, depending on what solution is used, there may be a runtime overhead involved.


\section{Haskell}

Haskell supports both parametric and ad-hoc polymorphism.

\subsection{Parametric Polymorphism}

Parametric polymorphism is achieved the same way it is in OCaml, that is, when writing a function that does not rely on its parameters' types, then one can define the function's type with type parameters.

% ===== removed to fit within the page limit
For example, the list length function:
\begin{minted}{haskell}
length :: [a] -> Int
length [] = 0
length (x:xs) = 1 + length xs
\end{minted}
where \verb|a| is the type parameter for the type of the list's elements.
% =====

\subsection{Ad-hoc Polymorphism: Type Classes}\label{haskell-type-classes}

According to the Haskell committee, there was no standard solution available for ad-hoc polymorphism when they designed the language, so \emph{type classes} were developed and judged successful enough to be included in the Haskell design~\cite{type-classes-original}.

In its simplest form, a type class is not much more than an interface as seen in object oriented languages, meaning it simply defines a named set of operations. For example, to define an interface for checking values for equality, one can define an \verb|Eq| type class with a class parameter \verb|a|, for which a \verb|(==)| function should exist:
\begin{minted}{haskell}
class Eq a where (==) :: a -> a -> bool
\end{minted}
% ===== removed to fit within the page limit
Mind that it's easy to misunderstand the \verb|a| class parameter as being similar to a type parameter used for parametric polymorphism, however this is not the case. Where with parametric polymorphism, \verb|a| would stand for \emph{any} type, in this case \verb|a| stands for a \emph{class} or \emph{set} of \emph{specific} types for which instances of \verb|Eq| exist.
% =====

The type class itself so far provides no functionality. We need to create an instance of the type class for a specific type, for example for \verb|Int|:
\begin{minted}{haskell}
instance Eq Int where (==) x y = eqInt x y
\end{minted}
Similarly, an instance of the \verb|Eq| type class can be created for the specific type \verb|Float|:
\begin{minted}{haskell}
instance Eq Float where (==) x y = eqFloat x y
\end{minted}
When we now try to use the \verb|(==)| function with parameters of types other than \verb|Int| or \verb|Float|, we will get a compile time error.

\subsubsection{Type constraints}\label{constraints}

So far, we have seen how we can use type classes to overload functions defined \emph{within} the type class, however we can also use them to define standalone functions with types implementing specific type classes.

Assume we want to define some kind of cost to our data types. We can use a type class to convey this idea:
\begin{minted}{haskell}
class Cost a where cost :: a -> Int
\end{minted}
A possible instance for some custom data type could then look like this:
\begin{minted}{haskell}
data Custom = Cheap | Expensive
instance Cost Custom where
  cost Cheap = 0
  cost Expensive = 1
\end{minted}
Now say we want to define a function \verb|cheapest| that takes a list of elements of a type that defines a \verb|cost|, and then return the cheapest element of that list. Such a function could look like this:
\begin{minted}{haskell}
cheapest :: Cost a => [a] -> a
cheapest [x] = x
cheapest (x:xs)
  | cost x <= cost y = x
  | otherwise = y
  where y = cheapest xs
\end{minted}
Note the function's type \verb|Cost a => [a] -> a|. It means that \verb|cheapest| is a function of type \verb|[a] -> a| for any type \verb|a| that is an instance of \verb|Cost|.

The similarities to parametric polymorphism are a lot more fitting in this case: \verb|a| for the \verb|cheapest| function is also a type parameter where the exact type does not matter for the function. The difference is that, whereas in the case of \verb|length|, any type can be inserted into \verb|a|, for \verb|cheapest| only types that implement a \verb|cost| can be used. This is necessary as the \verb|cost| function is used in the implementation of \verb|cheapest|.

\subsubsection{Creating instances from other instances}

One particular strength of type classes is that they allow one to create instances based on other existing type class instances. We extend the example from \autoref{constraints} to illustrate this: We are given a list of \verb|Custom|s and want to determine the overall sum of the list's costs. We could write an instance for \verb|Cost [Custom]|, however this seems rather verbose, considering we might want to have an implementation of this function for other Cost instances, too. Instead, we can define an instance of the \verb|Cost| type class for any list of type \verb|[a]| where \verb|a| has an instance for \verb|Cost|:
\begin{minted}{haskell}
instance Cost a => Cost [a] where
  cost = sum . map cost
\end{minted}
Now, Haskell will automatically create an instance for \verb|Cost [Custom]| the moment we create the instance for \verb|Cost Custom|.

This looks quite similar to type constraints and this is no coincidence. We are essentially doing a type constraint, only for the entire instance of a type class. We can see this as applying the type constraint to every function declared within the type class. As such, we can again use \verb|a|'s \verb|cost| function inside the \verb|Cost [a]| instance's functions.

% MAYBE mention scoping rules here, as done in Rust

% Type classes also support the notion of subclasses, however we will not cover this here as it exceeds the scope of this paper.
% MAYBE cover this anyway?

\subsubsection{Multiple Parameters}\label{haskell-add}

Standard Haskell does not allow for type classes to have more than one class parameter. However, modern compilers like \verb|ghc| support the \verb|-XMultiParamTypeClasses| compiler option that allows one to create type classes with multiple parameters.

The most obvious limitation this solves is the fact that, with single parameter type classes and the perspective of operator overloading, we can only define an \verb|add| function to add values of the same type. With multiple parameters, one could define an \verb|Add| type class that allows adding two values of different types.\footnote{This is not how the \verb|add| function is implemented in standard Haskell, but this is how it \emph{could} be done.}
\begin{minted}{haskell}
class Add a b where add :: a -> b -> a
\end{minted}
However this is still quite ugly as the return type of the addition is simply the type of the first argument, which need not always be what we want. An obvious fix to this would be adding a third parameter to the \verb|Add| type class:
\begin{minted}{haskell}
class Add a b c where add :: a -> b -> c
\end{minted}
While this works, it poses a strange issue: One can define multiple additions for the same type that have a different output. Let's say we define the following instances for the \verb|Add| type class with three parameters:
\begin{minted}{haskell}
instance Add Int Float Int where
  add x y = x + floatToInt y
instance Add Int Float Float where
  add x y = intToFloat x + y
\end{minted}
At first this might seem fine, however when one were to write \verb|add 1 3.2|, the actual return type of such an expression is not well defined anymore and needs to be annotated explicitly for every call, e.g. \verb|add 1 3.2 :: Float|. This is not very ergonomic, thus a different more elegant solution exists.

% https://downloads.haskell.org/~ghc/7.8.4/docs/html/users_guide/type-class-extensions.html#:~:text=7.6.2.2.2.%C2%A0Adding%20functional%20dependencies
% FUNCTIONAL DEPENDENCIES CAN ALSO FIX THIS POTENTIALLY

\subsubsection{Associated types}

% ONLY WORKS WITH TypeFamilies COMPILER OPTION

Associated Types are a way to define more class parameters, that are defined inside the instance of a type class and not in its signature. That means we can define \verb|Add| with three class parameters, where the third of those is fixed once the other two are.

In our \verb|Add| example, a solution using an associated type could look like this:
\begin{minted}{haskell}
class Add a b where
  type AddOutput a b
  add :: a -> b -> AddOutput a b
\end{minted}
We define that with any instance of \verb|Add a b| for specific types \verb|a| and \verb|b|, there shall also be a third type that's given the name \verb|AddOutput a b|. This type is then the output of our \verb|add| function. This concept ensures that there shall only be one well defined output type for an addition of two specific types \verb|a| and \verb|b|.

To define an addition for \verb|Int| and \verb|Float|, this means we have to decide what output we would even want in this case. Let's assume the choice was made to output a \verb|Float| in this case, the instance for the \verb|Add| type class would then look like this:
\begin{minted}{haskell}
instance Add Int Float where
  type AddOutput Int Float = Float
  add x y = intToFloat x + y
\end{minted}
One unfortunate limitation is that the \verb|AddOutput| name is not namespaced, i.e. we will need to have a different name for each associated type we have inside of different type classes.

\subsubsection{A matrix example}\label{haskell-matrix}

Let's assume we have a data type called \verb|Matrix a| which describes a matrix whose elements are of type \verb|a|. We can then use our previously defined \verb|Add| type class to define a matrix addition for any matrices whose elements' types are addable:
\begin{minted}{haskell}
instance Add a b => Add (Matrix a) (Matrix b) where
  type AddOutput (Matrix a) (Matrix b) =
    Matrix (AddOutput a b)
  add x y = -- ...
\end{minted}
The first line can be read as ``implement \verb|Add| for two matrices whose elements' types implement \verb|Add|''. The second and third lines mean ``the output of such an addition shall be another matrix whose elements' type is that of the output of adding the elements of the original two matrices together''.

Note that this is \textbf{not} how standard Haskell implements overloaded addition. Instead, Haskell has a \verb|Num| type class that defines multiple operators, such as \verb|(+)| and the unary negate, and only defines those operations on two elements \emph{of the same type}. We introduced our \verb|Add| type class merely to showcase multiple parameters and associated types.


\section{Rust}

Rust is known to be a language that combines many different features of many different programming languages into one language, with the goal of keeping the good and eliminating the bad with each addition.
When it comes to polymorphism, Rust combines the concept of type classes with generics, the latter of which is usually seen in object oriented languages like Java, C\# and C++.

\subsection{Generics: Rust's Parametric Polymorphism}

Generics, in their simplest form, are a form of parametric polymorphism. As such, the length function on slices can be defined with a generic which we will name \verb|T|.
\footnote{Because Rust only implements many ideas from functional languages, but is in itself a C-like imperative language and thus not purely functional, the terminology used for similar things will be slightly different, so will the syntax. In this case for example, we use a slice in place of a list. A slice in Rust is a reference to a specific area inside an array (or similar), and it can also be a reference to the entire array itself. The type \verb|usize| is one of Rust's unsigned integer types and will be used as such.}
\begin{minted}{rust}
fn len<T>(slice: &[T]) -> usize {
    // calculate and return length
}
\end{minted}
\footnote{Pay no mind to the abundance of ampersands inside this Rust code, it is not of importance for this topic.}
We will leave out the implementation of the length function here, because it requires knowledge of the underlying layout of slices. The function is given by Rust's STL~\cite{rust-len}. We will also slightly alter the declaration of this function. As it stands, \verb|len| is declared as a standalone function, but in the real Rust STL it is actually defined as a member function on the slice primitive. To implement member functions on types, Rust uses the \verb|impl| keyword. Note that we also must declare the generic \verb|T| with the \verb|impl| keyword, as the type we are implementing on itself uses the generic:
\begin{minted}{rust}
impl<T> [T] {
    fn len(&self) -> usize {
        // calculate and return length
    }
}
\end{minted}
\verb|self| is a keyword used in Rust to allow a call with the dot operator, e.g. so one can write \verb|my_slice.len()| instead of \verb|len(my_slice)|. It implicitly is of the type the \verb|impl| block is defined on.

\subsection{Traits: Rust's Type Classes}

In their simplest form, traits are the exact same concept as type classes, they simply define a named set of operations. This is because Rust's traits are heavily inspired by type classes~\cite{rust-reference}. The main difference between the two being a slight change in nomenclature. Whereas in Haskell we explicitly gave the type we are later instancing on a name, i.e. the \verb|a| in \verb|class Eq a|, in Rust this type is implicitly called \verb|Self|. \verb|Self| is not the same as \verb|self|, \verb|self| is a value and \verb|Self| is \verb|self|'s type.

To see how traits work, we will implement the same equality type class from \autoref{haskell-type-classes}:
\footnote{Rust's STL also defines an \verb|Eq| trait, however this is actually a marker trait without any function declarations. Instead, what we are doing here is more similar to the STL's \verb|PartialEq| trait (although that one also has a default implementation for not equals).}
\begin{minted}{rust}
trait Eq { fn eq(&self, other: &Self) -> bool; }
\end{minted}
What we called a type class instance in Haskell we call a trait implementation in Rust, because doing so extends the previously used \verb|impl| keyword:
\begin{minted}{rust}
impl Eq for usize {
    fn eq(&self, other: &Self) -> bool {
        // equality function for values of type usize
        eq_usize(self, other)
    }
}
\end{minted}
We can of course also implement the trait on other types, such as Rust's floating point primitive \verb|f32|:
\begin{minted}{rust}
impl Eq for f32 {
    fn eq(&self, other: &Self) -> bool {
        eq_f32(self, other)
    }
}
\end{minted}

Like with Haskell, Rust is also statically typed, and therefore calling \verb|eq| on either \verb|usize| or \verb|f32| types will automatically pick the correct function at compile time, and calling it on any other type will give us a compile time error.

\subsubsection{Trait bounds: Rust's type constraints}

Trait bounds allow us to tell the Rust compiler that a generic function only makes sense, if those generic types implement a specific trait. This is the exact same as Haskell's type constraints discussed in \autoref{constraints}. We will also use the same \verb|Cost| example and thus first define a \verb|Cost| trait:
\begin{minted}{rust}
trait Cost { fn cost(&self) -> usize; }
\end{minted}
Then we will implement the trait for some custom data type, which will be an enum with two variants with the exact same semantics as the custom data type we used for the Haskell example:
\begin{minted}{rust}
enum Custom { Cheap, Expensive }
impl Cost for Custom {
    fn cost(&self) -> usize {
        match self {
            Custom::Cheap => 0,
            Custom::Expensive => 1,
        }
    }
}
\end{minted}
And now we again define a \verb|cheapest| function, however this time we define it on slices instead of lists:
\begin{minted}{rust}
fn cheapest<T>(elements: &[T]) -> &T where T: Cost {
    // the exact implementation is not relevant
    elements.iter()
        .min_by(|x, y| x.cost().cmp(&y.cost()))
        .unwrap()
}
\end{minted}
Inside the \verb|cheapest| function, we use the \verb|cost| method on elements of the slice. To do so, we must tell the compiler that elements of the slice implement such function, which is done by giving the \verb|cheapest| function a \verb|where| clause. Inside a \verb|where| clause, we can define as many trait bounds as we want which are of the form \verb|type: bound|. In this case, we are adding a bound \verb|Cost| for the generic \verb|T| inside the \verb|where| clause, meaning that the function \verb|cheapest| shall only exist for \verb|T|s that have a \verb|Cost| implementation.

\subsubsection{Creating implementations from other implementations}

As with Haskell, we can create implementations for generic types, where the generics are constrained, i.e. have trait bounds. For this, we will again implement \verb|Cost| for slices where the slices' elements implement \verb|Cost|:
\begin{minted}{rust}
impl<T> Cost for [T] where T: Cost {
    fn cost(&self) -> usize {
        self.iter().map(|x| x.cost()).sum()
    }
}
\end{minted}
% Rust only allows us to do so if \textit{either} the type we're implementing on \textit{or} the trait is not foreign, i.e. local to the current crate (which is Rust's version of a package)~\cite{rust-book}. This is done to prevent certain situations that could cause ambiguities when selecting a method.
% TODO the above is not really necessary is it?

\subsubsection{Generic traits: Rust's multi-parameter type classes}

So far we faced a similar problem we did in Haskell: If we wanted to define an \verb|Add| trait with an \verb|add| function, that function would have to take the same types for both parameters. This limitation can be removed in Rust by using generics inside the trait's definition:
\begin{minted}{rust}
trait Add<T> { fn add(self, rhs: T) -> Self; }
\end{minted}
Unlike in Haskell, we did not have to enable a language extension first, as Rust supports this out of the box.

As in \autoref{haskell-add}, the output is set to be of the same type as the left hand side, which we can fix by adding a second generic:
\begin{minted}{rust}
trait Add<T, U> { fn add(self, rhs: T) -> U; }
\end{minted}
Which again poses the issue that an addition between two types does not have a well defined output type. We can implement addition between integers and floating point numbers with differing outputs:
\begin{minted}{rust}
impl Add<f32, usize> for usize {
    fn add(self, rhs: f32) -> usize {
        // ...
    }
}
impl Add<f32, f32> for usize {
    fn add(self, rhs: f32) -> f32 {
        // ...
    }
}
\end{minted}
If we were to call \verb|1_usize.add(3.2_f32)|, we would have to specify \textit{which} \verb|add| function we want to call:
\footnote{This uses Rust's Fully Qualified Syntax for Disambiguation~\cite{rust-book} and it looks rather complex. However, by simply binding the result to a variable, it would be enough to give said variable a type: \verb|let k: f32 = 1_usize.add(3.2_f32);|.}
\mint{rust}|<usize as Add<f32, f32>>::add(1_usize, 3.2_f32)|

\subsubsection{Associated types}

Once again, Rust also defines the notion of associated types, which also works very similar to Haskell.

Just as in Haskell, an associated type is another type parameter that is fixed for a specific configuration of types inserted in the generics. And as before, we will use an associated type to define the output type of an addition.

First we must edit the \verb|Add| trait's definition. We will add an associated type called \verb|Output| and while we're at it, we will also change the name of the generic \verb|T| to \verb|Rhs| and default it to \verb|Self|:
\footnote{This is actually how the real world \verb|Add| trait in Rust's STL is defined, with the exception of access modifiers.}
\begin{minted}{rust}
trait Add<Rhs = Self> {
    type Output;
    fn add(self, rhs: Rhs) -> Self::Output;
}
\end{minted}
The type \verb|Output| is now fixed for a given \verb|Self| and \verb|Rhs| type. Notice how we didn't call the associated type \verb|AddOutput| like we did in the Haskell example. The reason being that, in Rust, the associated type is in the namespace of the surrounding trait. That is why, to use the associated type as the output type of the \verb|add| function, we need to qualify it as \verb|Self::Output| which implicitly translates to \verb|<Self as Add<Rhs>>::Output| inside this trait. That way, we can reuse the \verb|Output| name for all operations.

Implementing the \verb|Add| trait now means one also has to define the \verb|Output| associated type, which we will set to be a float like we did with Haskell:
\begin{minted}{rust}
impl Add<f32> for usize {
    type Output = f32;
    fn add(self, rhs: f32) -> Self::Output {
        // ...
    }
}
\end{minted}
If we were to add another implementation with \\ \verb|type Output = usize|, the Rust compiler would detect the ambiguity and throw an error.

\subsubsection{A matrix example}

Rust generics can also be used for structs, which are Rust's custom data types. To mirror the matrix example from \autoref{haskell-matrix}, let us assume we have a generic matrix struct \verb|Matrix<T>| where the matrix's elements are of type \verb|T|.

First, we will define the addition over matrices with the same \verb|T|:
\begin{minted}{rust}
impl<T> Add for Matrix<T> where T: Add<Output = T> {
    type Output = Matrix<T>;
    fn add(self, rhs: Matrix<T>) -> Self::Output {
        // ...
    }
}
\end{minted}
Note that we do not have to specify the type of the right hand side, as it defaults to \verb|Self|, which is \verb|Matrix<T>| in this case. Note also that we had to bound \verb|T| to be addable with itself where the result is again a \verb|T|. The \verb|where| clause can thus be read as "implement only for types \verb|T| that are addable with itself and where such an addition returns a \verb|T|".

We extend this implementation for additions where the element type of the right hand side matrix (which we will call \verb|U|) is different to the element type of the left hand side matrix (\verb|T|). We also make the Output type of the addition of a \verb|T| and \verb|U| variable, by giving the output type the name \verb|O|:
\begin{minted}{rust}
impl<T, U, O> Add<Matrix<U>> for Matrix<T>
where
    T: Add<U, Output = O>,
{
    type Output = Matrix<O>;
    fn add(self, rhs: Matrix<U>) -> Self::Output {
        // ...
    }
}
\end{minted}
This is essentially the exact same as Haskell's matrix example.

% \subsubsection{Making the matrix's size known at compile time}

% As one last extension to our matrix example, we can make use of Rust's system for \textit{const generics}. These offer a limited form of dependent types in Rust, limited in the sense that the values for each generic must be known at compile time and cannot be changed at runtime.

% What this allows us to do is define matrix multiplication only for matrices of correct sizes. To demonstrate, let us first look at how the appropriate matrix struct would be defined:
% \begin{minted}{rust}
% struct Matrix<T, const M: usize, const N: usize> {
%     // ...
% }
% \end{minted}
% We can view this as a definition of matrices of type $\texttt{T}^{\texttt{M} \times \texttt{N}}$.

% To see how one would implement a matrix multiplication, we will first have to define a \verb|Mul| trait, which will look very similar to the \verb|Add| trait:
% \begin{minted}{rust}
% trait Mul<Rhs = Self> {
%     type Output;

%     fn mul(self, rhs: Rhs) -> Self::Output;
% }
% \end{minted}

% \begin{figure*}[h]
% \begin{minted}{rust}
% impl<Tlhs, Trhs, Tout, const M: usize, const N: usize, const L: usize> Mul<Matrix<Trhs, N, L>>
%     for Matrix<Tlhs, M, N>
% where
%     Tlhs: Mul<Trhs, Output = Tout>,
%     Tout: Sum,
% {
%     type Output = Matrix<Tout, M, L>;

%     fn mul(self, rhs: Matrix<Trhs, N, L>) -> Self::Output { /* ... */ }
% }
% \end{minted}
% \caption{Rust matrix example with const generics}
% \label{full-matrix}
% \end{figure*}

% \autoref{full-matrix} shows how we can implement a matrix multiplication.
% \footnote{Technically we would also need to add \verb|Copy| or \verb|Clone| trait bounds for an implementation to work, however for that we would need to talk about Rust's ownership system which is out of scope for this paper.}
% Reminder: matrix multiplication is a function
% \begin{align*}
% \texttt{Tlhs}^{\texttt{M} \times \texttt{N}} \times \texttt{Trhs}^{\texttt{N} \times \texttt{L}} \to \texttt{Tout}^{\texttt{M} \times \texttt{L}} : (A, B) \mapsto C \\
% \text{where } \forall i, j: c_{i, j} = \sum_{k = 1}^{\texttt{N}} a_{i, k} \cdot b_{k, j}
% \end{align*}.

% At first we define all generics used, which are \verb|Tlhs|, \verb|Trhs|, \verb|Tout| as well as all three const generics \verb|M|, \verb|N| and \verb|L|.
% Afterwards we can read the rest of the line as "implement multiplication between matrices of type $\texttt{Tlhs}^{\texttt{M} \times \texttt{N}}$ and $\texttt{Trhs}^{\texttt{N} \times \texttt{L}}$". The trait bound for \verb|Tlhs| is as expected, however note how we also have to define that \verb|Tout| needs to be sum-able in some way, with \verb|Sum| being another trait defining such functionality that we have not defined here.

% The defining strength here is that we define multiplication \textit{only} on matrices that have the correct size, and as that size also must be known at compile time, that means that we can also check if the matrices' sizes are correct at compile time. We can also deduce the resulting size of such an operation, all at compile time. To achieve this, in many languages one would have to manually define matrices of specific sizes, however in Rust we can do so completely generically and it will work with matrices of \textit{any} sizes.


\section{Ocaml}

As already noted before, Ocaml supports parametric polymorphism as one would expect. However the language has no concept of ad-hoc polymorhism, i.e. there is no way to overload functions. Nontheless, Ocaml inherits ML's powerful module system which still allows the language to express and convey complicated relations between types.

At first glance, Ocaml's module system might seem like a way to define modules, not interfaces. However these concepts aren't actually all that different in the end. Interfaces define a set of functions over either one or, depending on the language, multiple types. Modules do essentially the same.

Modules in Ocaml are also a collection of types and functions. Let's say we want to define a module for working with matrices of integers, one could write something like this:
\begin{minted}{ocaml}
module DenseMatrix =
  struct
    type elem = int
    type t = elem list list
    let add a b = (* ... *)
  end
\end{minted}
Obviously this Matrix module does not define all functions one would want in such a module, but we will leave out other functions for now.

Ocaml modules will also have an associated \textit{signature}. The signature includes information about the function names and types, the names of the defined types and optionally what those types are defined to. The reason why this latter part is optional is to have the ability to hide implementation details from the user of the module.

Ocaml can infer the signature of our \verb|DenseMatrix| module, but let's say we want to hide the fact that our Matrix uses an \verb|elem list list|, so we'll define the signature ourselves:
\begin{minted}{ocaml}
module type Matrix =
  sig
    type elem = int
    type t
    val add : t -> t -> t
  end

module DenseMatrix : Matrix =
  struct
    type elem = int
    type t = elem list list
    let add a b = (* ... *)
  end
\end{minted}
What this also allows us to do is create a seperate \verb|SparseMatrix| module with the same signature:
\begin{minted}{ocaml}
module SparseMatrix : Matrix =
  struct
    type elem = int
    type t = (* ... *)
    let add a b = (* ... *)
  end
\end{minted}

Thus, from the perspective of interfaces, module signatures can be seen as type classes and modules as instances~\cite{modular-type-classes}.

The main difference to type classes (and by extension the reason why Ocaml modules are not ad-hoc polymorphic) is that the \verb|add| function will be namespaced inside each module we defined. If we wanted to use the \verb|DenseMatrix|'s add function, we would have to specify it as \verb|DenseMatrix.add|. It is also possible to use the \verb|open| keyword to bring the entire module into the current scope, however this will only ever work for \textit{one} of these modules at a time. As such there is no logic behind automatically choosing the correct \verb|add| function for a given context, which is what ad-hoc polymorphism and overloading is about. Instead, the caller must specify an exact function they want to call.

In contrast to this, for Haskell, the implemented functions in a type class were available independently from specific instances, i.e. in Ocaml's terms, the module signature's defined functions could be used without specifying the exact module.

\subsection{Functors}

\subsubsection{Extending our matrices for more element types}

Let's take one step back and slightly alter our \verb|Matrix| module signature.
\begin{minted}{ocaml}
module type Matrix =
  sig
    type elem
    type t
    val add : t -> t -> t
  end
\end{minted}
In particular, we are not defining the type of \verb|elem| anymore. This allows us to implement multiple matrices for different types.

\begin{figure*}
\begin{minted}{ocaml}
module DenseIntMatrix : (Matrix with type elem = int) =
  struct
    type elem = int
    type t = elem list list
    let add a b = (* ... *)
  end

module SparseIntMatrix : (Matrix with type elem = int) =
  struct
    type elem = int
    type t = (* ... *)
    let add a b = (* ... *)
  end

module DenseFloatMatrix : (Matrix with type elem = float) =
  struct
    type elem = float
    type t = elem list list
    let add a b = (* ... *)
  end

module SparseFloatMatrix : (Matrix with type elem = float) =
  struct
    type elem = float
    type t = (* ... *)
    let add a b = (* ... *)
  end
\end{minted}
\caption{Ocaml matrix example with dense and sparse matrix}
\label{ocaml-all-matrix}
\end{figure*}

\begin{figure*}
\begin{minted}{ocaml}
module DenseMatrix (D : Addable) : (Matrix with type elem = D.t) =
  struct
    type elem = D.t
    type t = elem list list
    let add a b = (* implementation of Matrix addition using D.add *)
  end
\end{minted}
\caption{Ocaml functor matrix example}
\label{ocaml-matrix-functor}
\end{figure*}

\autoref{ocaml-all-matrix} shows possible dense and sparse matrix implementations for \verb|int|s and \verb|float|s. \verb|Matrix with type elem = int| simply means to replace \verb|type elem| with \verb|type elem = int| inside the \verb|Matrix| module signature, effectively exporting the type of \verb|elem|. This allows an outside user of a \verb|DenseIntMatrix| to know that \verb|elem| is an \verb|int|. This can be important if, let's say, we added a \verb|get| function that returns an \verb|elem| to get certain elements out of the matrix. If we didn't specify that \verb|elem| was of type \verb|int|, the user of our module would then not know that \verb|get| returns \verb|int|s, but rather some unknown and thus unusable type \verb|elem|.

Because we left out the implementations of these functions, it might not be immediately obvious, but the actual implementation of a dense matrix with \verb|float| values and a dense matrix with \verb|int| values will most likely look \textit{very} similar. The same is true for the sparse matrix. This is not exactly something we want, and this is where functors come in handy.

\subsubsection{Defining addition as a module}

Functors, as the name would suggest, are something rather similar to functions. In a sense, they can be seen as compile-time functions that take modules as arguments, and return other modules. They can also be seen as a sort of macro for module definitions.

A functor in our case could define the implementation of an entire \verb|DenseMatrix| for any type \verb|elem| that implements an addition. However, we don't have a concept of "implements an addition" in Ocaml yet. So let's define a module signature that conveys the concept of addition:
\begin{minted}{ocaml}
module type Addable =
  sig
    type t
    val add : t -> t -> t
  end
\end{minted}
Because our \verb|Matrix| module signature defines this very \verb|add| function, we can also define the \verb|Matrix| signature by including \verb|Addable|:
\begin{minted}{ocaml}
module type Matrix =
  sig
    type elem
    type t
    include Addable with type t := t
  end
\end{minted}
What \verb|include| does in Ocaml is to copy all definitions of another module/signature into the current module/signature. The reason why we don't use \verb|Addable|'s \verb|type t| but instead use our own is because we might want to add modules for multiplication etc. later on which should all share the same \verb|t|.

\subsubsection{Functors in action}

\autoref{ocaml-matrix-functor} shows how we can define a \verb|DenseMatrix| functor, that takes in an \verb|Addable| module as an argument called \verb|D|, and gives back a \verb|Matrix| module by implementing a dense matrix with elements of type \verb|D.t|. We can do the same for sparse matrices, but we will skip this here.

So far we can't use the \verb|DenseMatrix| functor, as we have yet to create a module with the \verb|Addable| signature, so let's do so for \verb|int|s:
\begin{minted}{ocaml}
module IntAddable : (Addable with type t = int) =
  struct
    type t = int
    let add a b = a + b
  end
\end{minted}

Now we can finally create a \verb|DenseIntMatrix| module by calling the \verb|DenseMatrix| functor on our \verb|IntAddable| module:
\begin{minted}{ocaml}
module DenseIntMatrix = DenseMatrix IntAddable
\end{minted}

We can now easily create a DenseMatrix for floating point values as well, simply by creating an \verb|Addable| module and calling the \verb|DenseMatrix| functor on it:
\begin{minted}{ocaml}
module FloatAddable : (Addable with type t = float) =
  struct
    type t = float
    let add a b = a +. b
  end

module DenseFloatMatrix = DenseMatrix FloatAddable
\end{minted}


\section{Conclusion}

Ocaml's module system is increadibly powerful and functors are a unique way of writing generic code. However, they do not have any concept of automatically choosing the right function for the current context. The user will always have to specify which function from which module they want exactly.

Another issue is that juggling functors, modules and module signatures can get very overwhelming and confusing. Whereas with type classes, we were able to create addition for matrices with varying outputs depending on the outputs of the additions for the matrices' element types, in Ocaml, building a generic matrix implementation even without the ability to have varying element types is already similarly complicated.

It is also rather verbose having to apply functors by hand. It is not possible for Ocaml to automatically apply functors and create unnamed modules out of them (although there are ideas on how to do so in limited cases~\cite{modular-type-classes}), and thus, to define a library for generic matrices, the library will only be able to offer a functor that the user must then apply for every type they want to use inside the matrices. It ends up only being one line of code, but it is still not as fluent as other solutions, not to mention one has to come up with a unique name for every one of the resulting modules.

On the other hand, one thing that neither Haskell nor Rust were able to do was to convey the generic idea of a \textit{matrix}. Both in Haskell and Rust, we were able to be generic over the elements of the matrix, \textbf{not} the implementation of it. In Ocaml, thanks to the fact that we can define a generic \verb|Matrix| module signature, wherein we can keep the format of the matrix undefined, we were also able to create functors for both dense and sparse matrices.

In Rust for example, defining a sparse matrix would mean to define a whole new separate struct with no correlation to the dense matrix struct. A general \verb|Matrix| trait would be difficult to define, as Rust's dynamic dispatch system\footnote{Dynamic dispatch is when the choice, which overloaded function is picked, happens at runtime and not compile time. In Rust this is done by using \textit{trait objects} and the \verb|dyn| keyword.} is rather verbose to work with (not to mention it's not idiomatic), and there is no way to define the structure of a struct from within a trait.

As such, it shall be mentioned that Ocaml modules aren't a direct equivalent to the notion of "Ocaml's type classes". In reality, Ocaml modules sit at a higher level. As the name suggest, whereas type classes and traits are essentially interface definitions for singular types, modules in Ocaml can include various types and define how they interact with one another. In that sense, modules in Ocaml sit on the same level of Rust's own modules. It's just with the existence of abstract module signatures as well as functors, that Ocaml modules are also found in the realm of abstract interfaces.

% ---- Bibliography ----
%\newpage{}
\bibliography{biblio}

% ---- Appendices ----
% \newpage
% \appendix
% \input{appendix.tex}

\end{document}
