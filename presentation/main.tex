%%
% This is an Overleaf template for presentations
% using the TUM Corporate Desing https://www.tum.de/cd
%
% For further details on how to use the template, take a look at our
% GitLab repository and browse through our test documents
% https://gitlab.lrz.de/latex4ei/tum-templates.
%
% The tumbeamer class is based on the beamer class.
% If you need further customization please consult the beamer class guide
% https://ctan.org/pkg/beamer.
% Additional class options are passed down to the base class.
%
% If you encounter any bugs or undesired behaviour, please raise an issue
% in our GitLab repository
% https://gitlab.lrz.de/latex4ei/tum-templates/issues
% and provide a description and minimal working example of your problem.
%%


\documentclass[
  english,            % define the document language (english, german)
  aspectratio=169,    % define the aspect ratio (169, 43)
  % handout=2on1,       % create handout with multiple slides (2on1, 4on1)
  % partpage=false,     % insert page at beginning of parts (true, false)
  % sectionpage=true,   % insert page at beginning of sections (true, false)
]{tumbeamer}


% load additional packages
\usepackage{booktabs}

\usepackage[outputdir=target]{minted}

\usepackage[style=verbose,backend=biber]{biblatex}
\addbibresource{biblio.bib}

% presentation metadata
\title[Comparing Type Classes and Traits]{Comparing Haskell's Type Classes to Rust's Traits (and OCaml's Modules)}
\subtitle{}
\author{Thomas Lindae}

\institute{\theChairName\\\theDepartmentName\\\theUniversityName}
\date[2022-07-21]{July 21\textsuperscript{st}, 2022}

\footline{\insertauthor~|~\insertshorttitle~|~\insertshortdate}


% macro to configure the style of the presentation
\TUMbeamersetup{
  title page   = TUM tower,
  part page    = TUM default,
  section page = TUM default,
  content page = TUM default,
  headline     = TUM threeliner,
  headline on  = {title page},
  footline     = TUM default,
  footline on  = {every page, title page=false},
  tower scale  = 1.0,
}

% available frame styles for title page, part page, and section page:
% TUM default, TUM tower, TUM centered,
% TUM blue default, TUM blue tower, TUM blue centered,
% TUM shaded default, TUM shaded tower, TUM shaded centered,
% TUM flags
%
% additional frame styles for part page and section page:
% TUM toc
%
% available frame styles for content pages:
% TUM default, TUM more space
%
% available headline options:
% TUM empty, TUM oneliner, TUM twoliner, TUM threeliner, TUM logothreeliner
%
% available footline options:
% TUM empty, TUM default, TUM infoline


\begin{document}

\maketitle

\section{Types of Polymorphism}

\begin{frame}[fragile]{Parametric Polymorphism}
\begin{quote}Parametric polymorphism [...] allows a single piece of code to be typed "generically," using variables in place of actual types, and then instantiated with particular types as needed. Parametric definitions are uniform: all of their instances behave the same.
\end{quote}~\fullcite[Chapter~23.2]{pierce-types}
\end{frame}

\begin{frame}[fragile]{Parametric Polymorphism in Haskell}
Length function for lists in Haskell:
\begin{minted}{haskell}
length :: [a] -> Int
length [] = 0
length (x:xs) = 1 + length xs
\end{minted}
\end{frame}

\begin{frame}[fragile]{Parametric Polymorphism in Rust}
Length function for slices in Rust:
\begin{minted}{rust}
fn len<T>(slice: &[T]) -> usize {
    // calculate and return length
}
\end{minted}

\pause \vspace{3mm}
With the \verb|impl| keyword:
\begin{minted}{rust}
impl<T> [T] {
    fn len(&self) -> usize {
        // calculate and return length
    }
}
\end{minted}
Usage: \mintinline{rust}{my_slice.len()}
\end{frame}

\begin{frame}[fragile]{Ad-hoc Polymorphism}
\begin{quote}
Ad-hoc polymorphism [...] allows a polymorphic value to exhibit different behaviors when "viewed" at different types. The most common example of ad-hoc polymorphism is overloading, which associates a single function symbol with many implementations; the compiler [...] chooses an appropriate implementation for each application of the function, based on the types of the arguments.
\end{quote}~\fullcite[Chapter~23.2]{pierce-types}
\end{frame}

\begin{frame}[fragile]{Ad-hoc Polymorphism}
Given \mintinline{haskell}{(+) :: Int -> Int -> Int} \\
and \mintinline{haskell}{(+) :: Float -> Float -> Float} \pause \vspace{3mm}

An expression such as \mintinline{haskell}{2.5 + 7.6} is easy. \pause \vspace{3mm}

However, a function such as
\begin{minted}{haskell}
add2 (x1, x2) (y1, y2) = (x1 + y1, x2 + y2)
\end{minted}
\pause has exponentially many different possible types: \vspace{3mm}

\begin{minted}{haskell}
add2 :: (Int, Int) -> (Int, Int) -> (Int, Int)
add2 :: (Int, Float) -> (Int, Float) -> (Int, Float)
add2 :: (Float, Int) -> (Float, Int) -> (Float, Int)
add2 :: (Float, Float) -> (Float, Float) -> (Float, Float)
\end{minted}
\end{frame}

\section{Type Classes}

\begin{frame}[fragile]{Type Classes in Haskell}
Definition of a type class:
\begin{minted}{haskell}
class Add a where
  (+) :: a -> a -> a
\end{minted}

\pause \vspace{3mm}

Instantiation of a type class:
\begin{minted}{haskell}
instance Add Int where
  (+) x y = addInt x y

instance Add Float where
  (+) x y = addFloat x y
\end{minted}

\pause \vspace{3mm}

Both \mintinline{haskell}{2 + 3} and \mintinline{haskell}{2.5 + 7.6} now work.
\end{frame}

\begin{frame}[fragile]{Type Classes in Rust (Traits)}
Definition of a trait:
\begin{minted}{rust}
trait Add {
    fn add(self, other: Self) -> Self;
}
\end{minted}

\pause \vspace{3mm}

Implementation of a trait:
\begin{minted}{rust}
impl Add for usize {
    fn add(self, other: Self) -> Self {
        add_usize(self, other)
    }
}
\end{minted}
\end{frame}

\section{Type Constraints}

\begin{frame}[fragile]{Type Constraints in Haskell}
Reminder:
\begin{minted}{haskell}
add2 (x1, x2) (y1, y2) = (x1 + y1, x2 + y2)
\end{minted}

\pause \vspace{3mm}

Typed with type constraints:
\begin{minted}{haskell}
add2 :: (Add a, Add b) => (a, b) -> (a, b) -> (a, b)
\end{minted}
\end{frame}

\begin{frame}[fragile]{Type Constraints in Rust (Trait Bounds)}
\begin{minted}{rust}
fn add2<T, U>((x1, x2): (T, U), (y1, y2): (T, U)) -> (T, U)
where
    T: Add,
    U: Add,
{
    (x1.add(y1), x2.add(y2))
}
\end{minted}
\end{frame}

\section{Creating Instances From Other Instances}

\begin{frame}[fragile]{Creating Instances From Other Instances in Haskell}
We can transform \mintinline{haskell}{add2} to its own \mintinline{haskell}{Add} instance.
\begin{minted}{haskell}
add2 :: (Add a, Add b) => (a, b) -> (a, b) -> (a, b)
add2 (x1, x2) (y1, y2) = (x1 + y1, x2 + y2)
\end{minted}

\pause \vspace{6mm}

\begin{minted}{haskell}
instance (Add a, Add b) => Add (a, b) where
  (+) (x1, x2) (y1, y2) = (x1 + y1, x2 + y2)
\end{minted}
\end{frame}

\begin{frame}[fragile]{Creating Implementations From Other Implementations in Rust}
\begin{minted}{rust}
impl<T, U> Add for (T, U)
where
    T: Add,
    U: Add,
{
    fn add(self, other: Self) -> Self {
        let (x1, x2) = self;
        let (y1, y2) = other;
        (x1.add(y1), x2.add(y2))
    }
}
\end{minted}
\end{frame}

\section{Multiple Class Parameters}

\begin{frame}[fragile]{Multiple Class Parameters in Haskell}
Using the \verb|-XMultiParamTypeClasses| compiler option allows for multiple class parameters:
\begin{minted}{haskell}
class Add a b c where
  add :: a -> b -> c
\end{minted}

\pause \vspace{6mm}

\begin{minted}{haskell}
instance Add Int Float Int where
  add x y = x + floatToInt y

instance Add Int Float Float where
  add x y = intToFloat x + y
\end{minted}
\end{frame}

\begin{frame}[fragile]{Multiple Class Parameters in Rust (Generic Traits)}
\begin{minted}{rust}
trait Add<T, U> {
    fn add(self, rhs: T) -> U;
}

impl Add<f32, usize> for usize {
    fn add(self, rhs: f32) -> usize {
        // ...
    }
}
\end{minted}
\end{frame}

\begin{frame}[fragile]{Multiple Class Parameters}
Problem: The return type of the operation is not well defined.

\vspace{6mm}

It is necessary to specify the return type for every call:
\begin{minted}{haskell}
(1 + 3.2) :: Float
\end{minted}

\pause \vspace{6mm}

Rust uses its Fully Qualified Syntax for Disambiguation:
\begin{minted}{rust}
<usize as Add<f32, usize>>::add(1_usize, 3.2_f32)
\end{minted}

\pause \vspace{3mm}

Or alternatively bind it to a variable:
\begin{minted}{rust}
let k: usize = 1_usize.add(3.2_f32);
\end{minted}
\end{frame}

\begin{frame}[fragile]{Multiple Class Parameters}
\begin{minted}{haskell}
class Add a b c where
  add :: a -> b -> c
\end{minted}

\vspace{3mm}

Solution: Make \verb|c| fully dependent on \verb|a| and \verb|b|.
\end{frame}

\section{Associated Types}

\begin{frame}[fragile]{Associated Types in Haskell}
Using the \verb|-XTypeFamilies| compiler option:
\begin{minted}{haskell}
class Add a b where
  type AddOutput a b
  add :: a -> b -> AddOutput a b
\end{minted}

\pause \vspace{6mm}

\begin{minted}{haskell}
instance Add Int Float where
  type AddOutput Int Float = Float
  add x y = intToFloat x + y
\end{minted}
\end{frame}

\begin{frame}[fragile]{Associated Types in Rust}
\begin{minted}{rust}
trait Add<Rhs = Self> {
    type Output;

    fn add(self, rhs: Rhs) -> Self::Output;
}
\end{minted}

\pause \vspace{3mm}

\begin{minted}{rust}
impl Add<f32> for usize {
    type Output = f32;

    fn add(self, rhs: f32) -> Self::Output { /* ... */ }
}
\end{minted}
\end{frame}

\section{A Matrix Example}

\begin{frame}[fragile]{A Matrix Example in Haskell}
Assume a type called \mintinline{haskell}{Matrix a}.

\pause \vspace{3mm}

\begin{minted}{haskell}
instance Add a b => Add (Matrix a) (Matrix b) where
  type AddOutput (Matrix a) (Matrix b) =
    Matrix (AddOutput a b)
  add x y = -- ...
\end{minted}
\end{frame}

\begin{frame}[fragile]{A Matrix Example in Rust}
Assume a type called \mintinline{rust}{Matrix<T>}.

\pause \vspace{3mm}

\begin{minted}{rust}
impl<Tlhs, Trhs, Tout> Add<Matrix<Trhs>> for Matrix<Tlhs>
where
    Tlhs: Add<Trhs, Output = Tout>,
{
    type Output = Matrix<Tout>;

    fn add(self, rhs: Matrix<Trhs>) -> Self::Output {
        // ...
    }
}
\end{minted}
\end{frame}

% \begin{frame}[fragile]{Matrices With Const Generics in Rust}
% \begin{minted}{rust}
% struct Matrix<T, const M: usize, const N: usize> {
%     // ...
% }

% trait Mul<Rhs = Self> {
%     type Output;

%     fn mul(self, rhs: Rhs) -> Self::Output;
% }
% \end{minted}
% \end{frame}

% \begin{frame}[fragile]{Matrices With Const Generics in Rust}
% \begin{minted}{rust}
% impl<Tlhs, Trhs, Tout, const M: usize, const N: usize, const L: usize> Mul<Matrix<Trhs, N, L>>
%     for Matrix<Tlhs, M, N>
% where
%     Tlhs: Mul<Trhs, Output = Tout>,
%     Tout: Sum,
% {
%     type Output = Matrix<Tout, M, L>;

%     fn mul(self, rhs: Matrix<Trhs, N, L>) -> Self::Output { /* ... */ }
% }
% \end{minted}
% \end{frame}

\end{document}
