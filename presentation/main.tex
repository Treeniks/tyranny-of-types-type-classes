%%
% This is an Overleaf template for presentations
% using the TUM Corporate Desing https://www.tum.de/cd
%
% For further details on how to use the template, take a look at our
% GitLab repository and browse through our test documents
% https://gitlab.lrz.de/latex4ei/tum-templates.
%
% The tumbeamer class is based on the beamer class.
% If you need further customization please consult the beamer class guide
% https://ctan.org/pkg/beamer.
% Additional class options are passed down to the base class.
%
% If you encounter any bugs or undesired behaviour, please raise an issue
% in our GitLab repository
% https://gitlab.lrz.de/latex4ei/tum-templates/issues
% and provide a description and minimal working example of your problem.
%%


\documentclass[
  english,            % define the document language (english, german)
  aspectratio=169,    % define the aspect ratio (169, 43)
  % handout=2on1,       % create handout with multiple slides (2on1, 4on1)
  % partpage=false,     % insert page at beginning of parts (true, false)
  % sectionpage=true,   % insert page at beginning of sections (true, false)
]{tumbeamer}


% load additional packages
\usepackage{booktabs}

\usepackage[outputdir=target]{minted}

% presentation metadata
\title[Comparing Type Classes and Traits]{Comparing Haskell's Type Classes to Rust's Traits (and OCaml's Modules)}
\subtitle{}
\author{Thomas Lindae}

\institute{\theChairName\\\theDepartmentName\\\theUniversityName}
\date[2022-07-21]{July 21\textsuperscript{st}, 2022}

\footline{\insertauthor~|~\insertshorttitle~|~\insertshortdate}


% macro to configure the style of the presentation
\TUMbeamersetup{
  title page   = TUM tower,
  part page    = TUM default,
  section page = TUM default,
  content page = TUM default,
  headline     = TUM threeliner,
  headline on  = {title page},
  footline     = TUM default,
  footline on  = {every page, title page=false},
  tower scale  = 1.0,
}

% available frame styles for title page, part page, and section page:
% TUM default, TUM tower, TUM centered,
% TUM blue default, TUM blue tower, TUM blue centered,
% TUM shaded default, TUM shaded tower, TUM shaded centered,
% TUM flags
%
% additional frame styles for part page and section page:
% TUM toc
%
% available frame styles for content pages:
% TUM default, TUM more space
%
% available headline options:
% TUM empty, TUM oneliner, TUM twoliner, TUM threeliner, TUM logothreeliner
%
% available footline options:
% TUM empty, TUM default, TUM infoline


\begin{document}

\maketitle

\section{Types of Polymorphism}
\begin{frame}[fragile]{Parametric Polymorphism}
\begin{minted}{haskell}
length :: [a] -> Int
length [] = 0
length (x:xs) = 1 + length xs
\end{minted}
\end{frame}

\begin{frame}[fragile]{Ad-hoc Polymorphism}
Given \mintinline{haskell}{(+) :: Int -> Int -> Int} \\ and \mintinline{haskell}{(+) :: Float -> Float -> Float}. \pause \vspace{3mm}

An expression such as \mintinline{haskell}{2.5 + 7.6} is easy. \pause \vspace{3mm}

However, a function such as
\begin{minted}{haskell}
add2 (x1, x2) (y1, y2) = (x1 + y1, x2 + y2)
\end{minted}
\pause has exponentially many different possible types: \vspace{3mm}

\begin{minted}{haskell}
add2 :: (Int, Int) -> (Int, Int) -> (Int, Int)
add2 :: (Int, Float) -> (Int, Float) -> (Int, Float)
add2 :: (Float, Int) -> (Float, Int) -> (Float, Int)
add2 :: (Float, Float) -> (Float, Float) -> (Float, Float)
\end{minted}
\end{frame}

\begin{frame}{Style setup}
  \texttt{tumbeamer} uses a concept of \textit{framestyles}, which are
  collections of beamertemplate definitions combined to get a certain style.
  Initially we defined 4 different framestyles for \texttt{title page},
  \texttt{part page}, \texttt{section page}, and \texttt{content page},
  respectively. \\
  The predefined options for every framestyle are summarized on a following
  slide. \\\vspace{3mm}

  With the keys \texttt{headline} and \texttt{footline} you can choose from
  predefined styles.

  Enabling or disabling of headlines and footlines can be done on a per
  framestyle basis with the keys \texttt{headline on} and
  \texttt{footline on}. As value you can pass a list of framestyles or the
  alias \texttt{every page}, which enables the headline or footline on all 4
  framestyles.
\end{frame}


\section{Second Section}
\begin{frame}{Frame of section two}
  \begin{enumerate}
    \item one
    \item two
    \item three
    \item four
  \end{enumerate}
\end{frame}

\end{document}
